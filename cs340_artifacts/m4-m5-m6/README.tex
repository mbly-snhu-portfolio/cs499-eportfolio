\documentclass[11pt]{article}
\usepackage[margin=1in]{geometry}
\usepackage{fontspec}
\setmainfont{IBM Plex Sans}
\setsansfont{IBM Plex Sans}
\setmonofont{IBM Plex Mono}
\usepackage{microtype}
\usepackage{titlesec}
\usepackage{enumitem}
\usepackage{hyperref}
\hypersetup{colorlinks=true, linkcolor=[rgb]{0.0,0.2,0.6}, urlcolor=[rgb]{0.0,0.2,0.6}, citecolor=[rgb]{0.0,0.2,0.6}}
\usepackage{xcolor}
\usepackage{listings}
\titleformat{\paragraph}[runin]{\bfseries}{}{}{}
\titlespacing*{\paragraph}{0pt}{0.5em}{0.6em}
\setlist{itemsep=2pt, topsep=4pt, parsep=0pt, partopsep=0pt}
\lstset{basicstyle=\ttfamily\small, breaklines=true, showstringspaces=false, aboveskip=0.3em, belowskip=0.3em}
\emergencystretch=2em

\title{AnimalShelter Project --- README}
\author{}
\date{\today}

\begin{document}
\maketitle

\section*{Overview}
AnimalShelter provides Create, Read, Update, and Delete (CRUD) access to the Austin Animal Center (AAC) dataset using MongoDB and PyMongo. The repository includes a robust CRUD class, a data importer with batching and duplicate avoidance, and a reproducible dashboard notebook featuring an interactive map, chart, and data table.

\section*{Features}
\begin{itemize}[leftmargin=*]
  \item CRUD class with clear contracts and input validation
  \item Data importer with duplicate detection, batching, and progress reporting
  \item Docker-based local environment with optional automated import
  \item Jupyter notebook dashboard (Dash + dash-leaflet + ECharts)
\end{itemize}

\section*{Architecture}
\begin{itemize}[leftmargin=*]
  \item \textbf{Data Layer}: MongoDB collection (default: \texttt{animals} in DB \texttt{aac}).
  \item \textbf{Access Layer}: \texttt{animal\_shelter/animal\_shelter.py} exposes a typed CRUD API.
  \item \textbf{Import Layer}: \texttt{animal\_shelter/data\_importer.py} loads, cleans, and batches CSV records.
  \item \textbf{Interface}: Jupyter notebook dashboard (map, chart, table) for exploration and reporting.
\end{itemize}

\section*{Repository Structure}
\begin{lstlisting}
m4-m5-m6/
├── animal_shelter/
│   ├── __init__.py
│   ├── animal_shelter.py
│   └── data_importer.py
├── assets/
│   └── aac_shelter_outcomes.csv (optional)
├── docs/
├── notebooks/
│   ├── ProjectTwoDashboard.ipynb
│   └── ModuleSixMilestone.ipynb
├── import_aac_data.py
├── docker-compose.yml
├── Dockerfile.importer
├── requirements.txt
└── README.md (original Markdown)
\end{lstlisting}

\section*{Data Model and Schema}
The importer normalizes column names to lowercase. Common fields in the AAC dataset include:
\begin{itemize}[leftmargin=*]
  \item \texttt{animal\_id} (string, expected unique), \texttt{name} (string)
  \item \texttt{animal\_type} (string), \texttt{breed} (string)
  \item \texttt{age\_upon\_outcome} (string), \texttt{outcome\_type} (string)
  \item Additional columns are preserved as-is (free-form schema).
\end{itemize}
Primary keys are not enforced at the database level. Deduplication is performed during import by dropping duplicate \texttt{animal\_id} entries.

\section*{System Requirements}
\begin{itemize}[leftmargin=*]
  \item Python: 3.10 or newer
  \item Docker and Docker Compose (optional, for local MongoDB)
\end{itemize}

\section*{Installation}
\begin{enumerate}[leftmargin=*]
  \item Create and activate a virtual environment.
  \begin{lstlisting}[language=bash]
python -m venv .venv
source .venv/bin/activate  # Windows: .venv\Scripts\activate
  \end{lstlisting}
  \item Install dependencies.
  \begin{lstlisting}[language=bash]
pip install --upgrade pip
pip install -r requirements.txt
  \end{lstlisting}
  \item (Optional) Start local services with Docker.
  \begin{lstlisting}[language=bash]
docker-compose up -d
  \end{lstlisting}
  If \texttt{assets/aac\_shelter\_outcomes.csv} is present, the importer runs automatically.
\end{enumerate}

\section*{Configuration}
Copy \texttt{env.example} to \texttt{.env} and update values as needed. Common variables include:
\begin{itemize}[leftmargin=*]
  \item \texttt{MONGODB\_HOST}, \texttt{MONGODB\_PORT}
  \item \texttt{AAC\_DATABASE}, \texttt{AAC\_COLLECTION}
  \item \texttt{AAC\_USER}, \texttt{AAC\_PASS}
  \item \texttt{BATCH\_SIZE}, \texttt{LOG\_LEVEL}
  \item \texttt{CSV\_SEARCH\_PATHS}
\end{itemize}

\subsection*{Environment Variables (Details)}
The application reads multiple environment variables with sensible fallbacks:
\begin{itemize}[leftmargin=*]
  \item Connection user/password: \texttt{MONGO\_USER}/\texttt{MONGO\_PASS} (fallbacks: \texttt{AAC\_USER}/\texttt{AAC\_PASS})
  \item Host/Port: \texttt{MONGO\_HOST}/\texttt{MONGO\_PORT} (fallbacks: \texttt{MONGODB\_HOST}/\texttt{MONGODB\_PORT}, defaults \texttt{localhost}:\texttt{27017})
  \item Database/Collection: \texttt{AAC\_DATABASE} (default \texttt{aac}), \texttt{AAC\_COLLECTION} (default \texttt{animals})
  \item Logging: \texttt{LOG\_LEVEL} (default \texttt{INFO})
  \item Importer batch size: \texttt{BATCH\_SIZE} (default \texttt{1000})
  \item CSV search paths: \texttt{CSV\_SEARCH\_PATHS} (comma-separated list; includes \texttt{./assets/aac\_shelter\_outcomes.csv} by default)
\end{itemize}

\subsection*{Sample .env}
\begin{lstlisting}[language=bash]
# Mongo credentials
AAC_USER=aacuser
AAC_PASS=changeme

# Mongo connection
MONGODB_HOST=localhost
MONGODB_PORT=27017
AAC_DATABASE=aac
AAC_COLLECTION=animals

# Importer
BATCH_SIZE=1000
CSV_SEARCH_PATHS=./assets/aac_shelter_outcomes.csv,./aac_shelter_outcomes.csv,./data/aac_shelter_outcomes.csv

# Logging
LOG_LEVEL=INFO
\end{lstlisting}

\section*{Usage}
\subsection*{CRUD Example}
\begin{lstlisting}[language=Python]
from animal_shelter import AnimalShelter

shelter = AnimalShelter()

animal_data = {
    "animal_id": "A123456",
    "name": "Buddy",
    "animal_type": "Dog",
    "breed": "Golden Retriever"
}

result = shelter.create(animal_data)
all_animals = shelter.read()
dogs = shelter.read({"animal_type": "Dog"})

shelter.close_connection()
\end{lstlisting}

\subsection*{Data Import}
\begin{lstlisting}[language=bash]
# Verify if data exists
python import_aac_data.py --check-only

# Import the dataset (skips if already present)
python import_aac_data.py

# Force re-import
python import_aac_data.py --force

# Tune batch size
python import_aac_data.py --batch-size 500
\end{lstlisting}

\paragraph{CSV Locations} Place \texttt{aac\_shelter\_outcomes.csv} in \texttt{assets/} to enable automatic import when using Docker. The importer also searches the current working directory and other common paths.

\subsection*{Importer CLI}
The repository includes a convenience script for import orchestration.
\begin{lstlisting}[language=bash]
# Check only (no import)
python import_aac_data.py --check-only

# Import if missing (skips when data already exists unless forced)
python import_aac_data.py

# Force reimport and use a custom batch size
python import_aac_data.py --force --batch-size 500
\end{lstlisting}
Behavior:
\begin{itemize}[leftmargin=*]
  \item Detects existing data and skips unless \texttt{--force} is specified.
  \item Cleans data (lowercases columns, drops empty/duplicate \texttt{animal\_id}).
  \item Imports in batches with a progress bar and aggregates success/failure stats.
  \item Verifies import with collection stats and record samples.
\end{itemize}

\subsection*{Indexes and Performance}
Call \texttt{AnimalShelter.ensure\_indexes()} to optimize common queries. Created indexes:
\begin{itemize}[leftmargin=*]
  \item \texttt{animal\_id}, \texttt{animal\_type}, \texttt{breed}, \texttt{outcome\_type}, \texttt{age\_upon\_outcome}
\end{itemize}
Guidelines:
\begin{itemize}[leftmargin=*]
  \item Prefer projections when reading large datasets.
  \item Use server-side filters (\texttt{read({...})}) to minimize transfers.
  \item Tune \texttt{BATCH\_SIZE} for import throughput; 500--2000 is a good starting range.
\end{itemize}

\section*{Dashboard}
Open the notebook and run all cells to start the Dash app in external mode.
\begin{lstlisting}[language=bash]
jupyter notebook notebooks/ProjectTwoDashboard.ipynb
\end{lstlisting}
Capabilities:
\begin{itemize}[leftmargin=*]
  \item Interactive filters for rescue categories (All, Water Rescue, Mountain/Wilderness, Disaster/Tracking).
  \item Data table with sorting, pagination, and row selection driving the map.
  \item Map (dash-leaflet) with base layer switching, clustering, auto-fit bounds, and scale control.
  \item Chart (ECharts) supporting bar, pie, and treemap visualizations of top breeds.
\end{itemize}

\section*{Docker Usage}
This repository includes a Compose file for a local stack:
\begin{itemize}[leftmargin=*]
  \item \textbf{mongodb}: MongoDB 7 with persistent volume and optional init script.
  \item \textbf{mongo-express}: Admin UI on \texttt{http://localhost:8081} (basic auth enabled in compose).
  \item \textbf{data-importer}: One-shot container that runs the importer at startup.
\end{itemize}
\begin{lstlisting}[language=bash]
docker-compose up -d        # start MongoDB and mongo-express
# (optional) run importer container once or re-run as needed
docker-compose up --build data-importer
\end{lstlisting}
Ensure \texttt{.env} is populated (see sample above). The importer container mounts \texttt{./assets} read-only.

\section*{API Reference (Python)}
\subsection*{Class: AnimalShelter}
\paragraph{Constructor}
\begin{lstlisting}[language=Python]
from typing import Optional

AnimalShelter(user: Optional[str] = None,
              password: Optional[str] = None,
              host: Optional[str] = None,
              port: Optional[int] = None)
\end{lstlisting}
\begin{itemize}[leftmargin=*]
  \item Parameters fall back to environment variables.
  \item Connection is validated using an admin ping.
\end{itemize}

\paragraph{create}
\begin{lstlisting}[language=Python]
create(data: dict) -> bool
\end{lstlisting}
Insert a single document; raises on invalid input.

\paragraph{read}
\begin{lstlisting}[language=Python]
read(criteria: dict | None = None) -> list[dict]
\end{lstlisting}
Return all documents or those matching \texttt{criteria}.

\paragraph{read\_by\_id}
\begin{lstlisting}[language=Python]
read_by_id(animal_id: str) -> dict | None
\end{lstlisting}
Return a single document by \texttt{animal\_id}.

\paragraph{update}
\begin{lstlisting}[language=Python]
update(criteria: dict, update_values: dict, many: bool = False) -> int
\end{lstlisting}
Apply MongoDB update operators (e.g., \texttt{\$set}); returns modified count.

\paragraph{delete}
\begin{lstlisting}[language=Python]
delete(criteria: dict, many: bool = False) -> int
\end{lstlisting}
Delete matching documents; returns deleted count.

\paragraph{get\_collection\_stats}
\begin{lstlisting}[language=Python]
get_collection_stats() -> dict
\end{lstlisting}
Return basic stats (document count, connection details).

\paragraph{ensure\_indexes}
\begin{lstlisting}[language=Python]
ensure_indexes() -> None
\end{lstlisting}
Create recommended indexes; safe to call multiple times.

\paragraph{close\_connection}
\begin{lstlisting}[language=Python]
close_connection() -> None
\end{lstlisting}
Close the underlying MongoDB client.

\section*{Testing}
\begin{lstlisting}[language=bash]
python -m pytest test_animal_shelter.py -v
\end{lstlisting}

\section*{Troubleshooting}
\begin{itemize}[leftmargin=*]
  \item \textbf{pandas on Python 3.13}: use the provided constraints file.
  \begin{lstlisting}[language=bash]
pip install -r requirements-py313.txt
  \end{lstlisting}
  \item \textbf{CSV not found}: ensure \texttt{assets/aac\_shelter\_outcomes.csv} exists or provide a path via arguments or environment.
  \item \textbf{MongoDB connection}: verify host, port, and credentials; ensure Docker services are running when applicable.
  \item \textbf{Authentication failed}: confirm \texttt{AAC\_USER}/\texttt{AAC\_PASS} and that the database exists.
  \item \textbf{Windows paths}: use escaped backslashes or forward slashes in paths passed to Python.
\end{itemize}

\section*{FAQ}
\begin{description}[leftmargin=*, labelindent=0em]
  \item[Can I import a subset of columns?] Yes. Preprocess the CSV before import; extra columns are preserved.
  \item[How do I re-run the importer safely?] Use \texttt{--force}. Consider clearing the collection first or using upserts if you enforce unique indexes.
  \item[Where do I change the default search paths?] Set \texttt{CSV\_SEARCH\_PATHS} in \texttt{.env} as a comma-separated list.
\end{description}

\section*{References}
\begin{itemize}[leftmargin=*]
  \item \href{https://www.mongodb.com/docs/}{MongoDB Documentation}
  \item \href{https://pymongo.readthedocs.io/}{PyMongo Documentation}
  \item \href{https://dash.plotly.com/}{Dash}
  \item \href{https://echarts.apache.org/}{Apache ECharts}
  \item \href{https://jupyter.org/}{Jupyter}
\end{itemize}

\vspace{1em}
\noindent\textit{Typeset with IBM Plex. Compiled with XeLaTeX.}

\end{document}


